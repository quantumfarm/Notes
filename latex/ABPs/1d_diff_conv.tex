\documentclass{../Templetes/1c}
\usepackage{graphicx}
\usepackage{bm}

\begin{document}
\title {1D Diffusion-convection Equation} 
\maketitle


\section{Conclusion}
\textcolor{red}{
In 1D, there is no steady solution of Eq. \eqref{s}!. Mathematically, 
the no flux boundary condition Eq. \eqref{b2} can not be satisfied!
Physically, because in the far away, there is always injecting flux, and in 1D the flux 
has no way to go but has to pass through the wall. In another word, the number in the wall will 
always accumulate! (Imagine inject water from a pipe continuely!)
}

\textcolor{red}{
In 2D, the inject particle can creep around the finite size wall, the no flux boundary could be satisfied. 
}



\section{Motivation}
To understand how does convection flow affect the interaction force in the fast rotation limit.



\section{Equation}
The number density current is 

\begin{equation}
j(x,t)=-D\pd_x g+(-\frac{V'}{\gamma}-u)g
\end{equation}

The conservation of particle number requires that

\begin{equation} \label{c}
\pd_t g+ \pd_x j=0
\end{equation}

In steady state $\pd_tg=0$, and eq.\eqref{c} is reduced to

\begin{equation} \label{s}
\pd_x^2 g+\pd_x\big[(\frac{V'}{\gamma}+u)g\big]=0,
\end{equation}  

where we have set $D=\gamma=1$ for simplicity. 

In addition, We assume the potential force half-harmonic, i.e. 
\begin{equation}
    V(x)=
   \begin{cases}
    \frac{K}{2}x^2, &\quad x<0, \\
    & \\
    0 & \quad x\geq 0 
   \end{cases}
  \end{equation}

\subsection{Boundary condition}
In the far away field, the density should be not affected by the wall, i.e.

\begin{equation}\label{b1}
  g(x\rightarrow\infty)=1
\end{equation}

And on the wall, there should be no flux (otherwise, if there is non-zero flux passing through the wall, 
then the particles will accumulate in the wall.)

\begin{equation} \label{b2}
j(x=0)=\Big(-\pd_x g-ug\Big)_{x=0}=0
\end{equation}

\section{Solution}




\subsection{Direct solution?}

\subsubsection{$x>0$}
For $x>0$, $V'=0$, eq.\eqref{s} is 

\begin{equation} \label{d}
\big[
g'+ug
\big]'=0
\end{equation}

So $g'+ug=C$. Using boundary condition (BC) $g(x\rightarrow \infty)=1$, 
$(g'(x\rightarrow\infty)=0$), we get $C=u$ and 

\begin{equation} \label{d2}
g'+ug=u
\end{equation}

The general solution is
\begin{equation} \label{d3}
g=Ae^{-ux}+1
\end{equation}
one can easily convince that eq.\eqref{d3} is the solution of eq.\eqref{d}. 
$A$ should be dertermined by the wall boundary condition eq.\eqref{b1}.

\begin{equation}
    j(x)=-uA^{-ux}+u(Ae^{-ux}+1)=u. 
\end{equation}

Clearly, the solution \label{d3} does not satisfy the no flux BC \eqref{b1}!

\subsubsection{$x<0$}
For $x<0$, eq.\eqref{s} is 

\begin{equation} \label{e}
\big[
g'+(V'+u)g
\big]'=0
\end{equation}

So
\begin{equation} \label{e2}
g'+(V'+u)g=C
\end{equation}
Let $g=e^{-V}e^{-ux}w(x)$, so $g'=-V'g-ug+e^{-V-ux}w'$, and Eq.\eqref{e2} is transformed to 

\begin{equation}
   e^{-V-ux}w'=C
\end{equation}

The general solution can be written as

\begin{equation}
   w=C_1\int_x^0 e^{V+ux'}dx'+C_2 
\end{equation}

Considering the BC $g(0)=A$, we get $C_2=A$. So the solution is

\begin{equation}
    g=e^{-V-ux} (C\int_x^0 e^{V+ux'}dx')+A
\end{equation}  


\end{document}


